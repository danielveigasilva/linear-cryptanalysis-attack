\documentclass{article}\title{Criptoanálise Linear}\author{Daniel Veiga da Silva Antunes }\date{Junho 2023}\begin{document}\maketitle

\section{Tabela de Bias}
A partir da análise da S-BOX disponibilizada geramos a Tabela de Bias abaixo.
\begin{center}
Tabela de Bias para SBOX (fator de 1/256)*
    \begin{tabular} {|c|c|c|c|c|c|c|c|c|c|c|c|c|c|c|c}
        \hline
        & \textbf{1} & \textbf{2} & \textbf{3} & \textbf{4} & \textbf{5} & \textbf{6} & \textbf{7} & \textbf{8} & \textbf{9} & \textbf{a} & \textbf{b} & \textbf{c} & \textbf{d} & \textbf{e} & \textbf{...}\\
        \hline
       \textbf{1} & -2 & -6 & -8 & -14 & -4 & 0 & -6 & -8 & -10 & -2 & -4 & 2 & 4 & -4 &  . . .\\
        \hline
       \textbf{2} & -12 & -4 & -8 & -10 & -10 & -2 & -10 & 0 & 16 & -16 & 0 & 2 & 14 & -2 &  . . .\\
        \hline
       \textbf{3} & -18 & -10 & 4 & 4 & 18 & 2 & 0 & 4 & -10 & 2 & 12 & 12 & 6 & 2 &  . . .\\
        \hline
       \textbf{4} & -12 & 16 & -4 & -6 & -10 & 10 & -10 & 2 & -10 & -10 & 2 & -4 & 0 & 0 &  . . .\\
        \hline
       \textbf{5} & 6 & 2 & -8 & -4 & -2 & 2 & -4 & -6 & 0 & 4 & -6 & -2 & 0 & 12 &  . . .\\
        \hline
       \textbf{6} & 0 & 0 & 0 & -4 & 8 & 8 & -4 & -2 & 18 & -2 & 10 & -2 & 6 & -6 &  . . .\\
        \hline
       \textbf{7} & -2 & 2 & 0 & -14 & 8 & -4 & -6 & 2 & -4 & 0 & -14 & 0 & -14 & -10 &  . . .\\
        \hline
       \textbf{8} & 2 & -4 & 10 & -14 & 0 & -6 & 4 & 8 & 10 & 20 & 2 & -2 & 4 & 6 &  . . .\\
        \hline
       \textbf{9} & 12 & 2 & 2 & 4 & 8 & -2 & -10 & 0 & -12 & -10 & -2 & 8 & 4 & -10 &  . . .\\
        \hline
       \textbf{a} & -14 & 4 & 2 & 0 & 2 & 4 & 2 & 0 & -10 & 8 & 2 & 0 & -2 & 8 &  . . .\\
        \hline
       \textbf{b} & 16 & 10 & 6 & -2 & 2 & -4 & -4 & -4 & 0 & -10 & -2 & 10 & 18 & -8 &  . . .\\
        \hline
       \textbf{c} & 14 & -4 & 6 & -12 & -2 & -20 & -6 & 10 & 0 & 2 & 4 & 2 & 4 & 6 &  . . .\\
        \hline
       \textbf{d} & 4 & 2 & 2 & -2 & 2 & -8 & 16 & -6 & -2 & 4 & -12 & 12 & -8 & -2 &  . . .\\
        \hline
       \textbf{.} &  . &  . &  . &  . &  . &  . &  . &  . &  . &  . &  . &  . &  . &  . & \\
\textbf{.} &  . &  . &  . &  . &  . &  . &  . &  . &  . &  . &  . &  . &  . &  . & \\
    \end{tabular}
\end{center}
* Tabela completa disponível no anexo biastable.csv

\section{Equações}
Em seguida localizamos os maiores bias em módulo e selecionamos as fórmulas relativas a valores de X com peso de hamming = 1, afim de isolarmos um único K. Com isto obtemos 8 equações:

\newblock

  Para X = 128 (hamming: 1) e Y = 205 com um bias de 22 temos:
\[ Pr[X_1 \oplus Y_1 \oplus Y_2 \oplus Y_5 \oplus Y_6 \oplus Y_8 = 0] = 0.586\]
\[K_1 = W_1 \oplus Y_1 \oplus Y_2 \oplus Y_5 \oplus Y_6 \oplus Y_8\]

  Para X = 64 (hamming: 1) e Y = 110 com um bias de -28 temos:
\[ Pr[X_2 \oplus Y_2 \oplus Y_3 \oplus Y_5 \oplus Y_6 \oplus Y_7 = 1] = 0.609\]
\[K_2 = W_2 \oplus Y_2 \oplus Y_3 \oplus Y_5 \oplus Y_6 \oplus Y_7 \oplus 1\]

  Para X = 32 (hamming: 1) e Y = 77 com um bias de 24 temos:
\[ Pr[X_3 \oplus Y_2 \oplus Y_5 \oplus Y_6 \oplus Y_8 = 0] = 0.594\]
\[K_3 = W_3 \oplus Y_2 \oplus Y_5 \oplus Y_6 \oplus Y_8\]

  Para X = 16 (hamming: 1) e Y = 69 com um bias de 26 temos:
\[ Pr[X_4 \oplus Y_2 \oplus Y_6 \oplus Y_8 = 0] = 0.602\]
\[K_4 = W_4 \oplus Y_2 \oplus Y_6 \oplus Y_8\]

  Para X = 8 (hamming: 1) e Y = 99 com um bias de -20 temos:
\[ Pr[X_5 \oplus Y_2 \oplus Y_3 \oplus Y_7 \oplus Y_8 = 1] = 0.578\]
\[K_5 = W_5 \oplus Y_2 \oplus Y_3 \oplus Y_7 \oplus Y_8 \oplus 1\]

  Para X = 4 (hamming: 1) e Y = 94 com um bias de 22 temos:
\[ Pr[X_6 \oplus Y_2 \oplus Y_4 \oplus Y_5 \oplus Y_6 \oplus Y_7 = 0] = 0.586\]
\[K_6 = W_6 \oplus Y_2 \oplus Y_4 \oplus Y_5 \oplus Y_6 \oplus Y_7\]

  Para X = 2 (hamming: 1) e Y = 255 com um bias de 26 temos:
\[ Pr[X_7 \oplus Y_1 \oplus Y_2 \oplus Y_3 \oplus Y_4 \oplus Y_5 \oplus Y_6 \oplus Y_7 \oplus Y_8 = 0] = 0.602\]
\[K_7 = W_7 \oplus Y_1 \oplus Y_2 \oplus Y_3 \oplus Y_4 \oplus Y_5 \oplus Y_6 \oplus Y_7 \oplus Y_8\]

  Para X = 1 (hamming: 1) e Y = 30 com um bias de -28 temos:
\[ Pr[X_8 \oplus Y_4 \oplus Y_5 \oplus Y_6 \oplus Y_7 = 1] = 0.609\]
\[K_8 = W_8 \oplus Y_4 \oplus Y_5 \oplus Y_6 \oplus Y_7 \oplus 1\]

\section{Resultados}

Foram executadas 4000000 interações para que o seguinte valor de chave fosse obtido:

\[ K = \{
 7, 6, 11, 2\} \]
\end{document}